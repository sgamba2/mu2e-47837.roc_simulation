\section {Introduction to the analysis}

In this note, we present initial results of the tracker DAQ commissioning.
\section{Description of teststand setup}
  The test stand included one DRAC card \cite{DRAC} and one DTC connected to the DAQ computer,
  96 channels in total.
  The ROC was operated in the data readout mode, and the digi FPGAs were pulsed by the internal pulser.
  The pulser has two different frequencies,  31.29 MHz/(2$^7$+1), or approximately 250 kHz,
  and 31.29 MHz/(2$^9$+1), or approximately 60 kHz.
  The event window, that simulates the distance between proton pulses, could be varied by us.
  The ROC firmware has an internal hit buffer which can store up to 255 hits.
  That should be sufficient for the data taking.
  Testing therefore could proceed in two different modes:
  \begin{itemize}
  \item
    The event window is large enough , so the total number of generated hits is greater than 255. In this case
    the ROC hit buffer always gets filled up, and only the first 255 hits are read out;
  \item
    The total number of hits within the event window is less than 255.
    In this case the ROC hit buffer doesn't get filled up and the total number of hits
    can vary from one event to another.
  \end{itemize}

\section{Monte Carlo simulation}\label{MonteCarlo}

To ensure a comprehensive understanding of our system, we initiated a Monte Carlo Simulation of our Data Acquisition (DAQ) system. 
We simulated performances of the readout logic that is purely digital, relying on different assumptions.
Given that the maximum allowable number of hits per event is 255, the simulation protocol adheres to the following sequential steps:
\begin{itemize}
  \item Within a time interval ranging from 0 $\mu$s to the reciprocal of the generator frequency, we generated the first event ($t_0$) by following a uniform distribution;
    \item After verified that hits from the same channel, are separated by an interval equivalent to the reciprocal of the generator frequency, we proceeded to 
      generate the second event ($t_1$) by summing this specific time increment to $t_0$;
      \item Subsequently, we verified whether the second pulse remained within the predefined event window. If it did, we included this hit in the ongoing event under construction;
      \item The process of event creation continued until the count of hits reached the maximum threshold of 255, at which point the event construction was terminated.
\end{itemize}
Furthermore, the simulation accounts for the channel to channel and FPGA to FPGA delays. 
This comprehensive approach ensures a faithful representation of the real-world operational aspects of the DAQ system.


%%% Local Variables:
%%% mode: latex
%%% TeX-master: t
%%% End:

