
\section{Overflow mode: RUN281}
%%%%%%%%%%%%%%%%%%%%%%%%%%%%%%%%%%%%%%%%%%%%%%%%%%%%%%%%%%%%%%%%%%%%%%%%%%%%%%
\subsection{Time distribution and Occupancy}\label{over}
The first distribution to look at is the time distribution and it is shown in Fig.\ref{fig:1}.
The left one is uniform and it comes from channel 0 of the first FPGA, however the right one, that represents the time distribution of channel 2 of the second FPGA, looks non trivial.

\begin{figure}[H]
  \hspace{-0.5in}
  \begin{tikzpicture}
    \node[anchor=south west,inner sep=0] at (0,0.) {
      % \node[shift={(0 cm,0.cm)},inner sep=0,rotate={90}] at (0,0) {}
      % \makebox[\textwidth][c] {
      \includegraphics[width=0.5\textwidth]{figures/pdf/figure_00007_timedistr_roc_simulation_ch0_281}
      % }
    };
    \node[anchor=south west,inner sep=0] at (10,0.) {
      % \node[shift={(0 cm,0.cm)},inner sep=0,rotate={90}] at (0,0) {}
      % \makebox[\textwidth][c] {
      \includegraphics[width=0.5\textwidth]{figures/pdf/figure_00003_timedistr_roc_simulation_ch2_281}
      % }
    };
  \end{tikzpicture}
  \caption{
    \label{fig:1}
    right: First FPGA's channel time distribution, left: Second FPGA's channel time distribution.
  }
\end{figure}

The initial though  was the occurrence of an interruption in data acquisition for specific channels at a certain time. To understand the differences between the two distibutions we plot the occupancy, total number of hits versus channel number, in Fig.\ref{fig:2}. This revealed a non uniform distribution. We compared with the Monte Carlo occupancy. Channels are ordered as the filling of channels occurred.
\begin{figure}[!h]
\centering
\includegraphics[width =0.8\textwidth]{figures/pdf/figure_00004_nhitsvschannel_roc_simulation_281}
\caption{Occupancy: number of hits versus channel. The ordering of channels adheres to the sequence prescribed by the Monte Carlo simulation.}
\label{fig:2}
\end{figure}
Actually, in the first plots that were made, channels were arranged in an ascending order, spanning from channel 0 to channel 95.
As second step, the occupancy distribution was plotted with an alternative channel order, as we would expect the filling to occur.
Our expectation was to observe a declining trend in the number of hits per channel, primarily due to the progressive filling of the buffer. 
However, contrary to our initial hypothesis, we encountered a distinctive pattern in which the number of hits exhibited a decrease until reaching zero hits at a specific channel, followed by a subsequent increase beyond the 72nd channel, which corresponds to channel 95.
It was deduced that the initial given channel ordering was incorrect. 
The first and the second lanes of the second FPGA were inverted and also some channels of the second FPGA.
Consequently, we have identified a revised channel sequence, as it is shown in Fig.\ref{fig:2}, verified also by the Monte Carlo simulation, which is as follows:
\begin{center}
\begin{tabular}{cc}
\textbf{FIST FPGA}: & \\
&91,85,79,73,67,61,55,49,\\
\textit{lane 1}: &43,37,31,25,19,13,7,1,\\
&90,84,78,72,66,60,54,48,\\
& \\
&42,36,30,24,18,12,6,0,\\
\textit{lane 2}: &93,87,81,75,69,63,57,51,\\
&45,39,33,27,21,15,9,3,\\
\textbf{SECOND FPGA}:&\\
&\\
&38,44,5,11,17,23,29,35,\\
\textit{lane 1}:&41,92,2,8,14,20,26,32,\\
&86,80,74,68,62,56,50,47,\\
 & \\
&95,89,83,22,16,28,34,40,\\
\textit{lane 2}:&46,53,59,65,71,77,10,4,\\
&94,88,82,76,70,64,58,52\\
\end{tabular}
\end{center}   
We can see a perfect adherence between data and simulation. We can interpret the occupancy plot, in Fig.\ref{fig:2}, as it follows: the initial group of channels in the histogram corresponds to the ones in which 4 hits (associated with the first FPGA) and 3 hits (associated with the second FPGA) are stored. 
Subsequently, a contrasting group of channels where the pattern reverses can be observed, 3 hits from the first FPGA followed by 4 hits from the second FPGA.
We can observe a little jump in the middle and it is due to channel to channel time differences.
Time differences between channels are three order of magnitude less than the FPGA ones, so these little jumps are very few, in our case only one appears.
Lastly, the concluding cluster of channels is comprised of those collecting 3 hits from the first FPGA and 3 hits from the second FPGA.
Coming back to Fig.\ref{fig:1}, some channels are always read and some others no, due to the fact that the ROC hit buffer gets filled up and only the first 255 hits are read out. This results in uniform time distribution for the first channels readout and in a non unform time distribution for the last readout channels, depending on $T_{gen}$ and $T_{EW}$. 


\subsection{Number of hits}
In conclusion, as in Fig. \ref{fig:3}, a key characteristic of the overflow mode becomes evident: the number of hits is concentrated at the maximum value of 255 hits. Consequently, this results in all events possessing the same dimension.
\begin{figure}[!h]
\centering
\includegraphics[width =0.8\textwidth]{figures/pdf/figure_00008_nhits_281}
\caption{Number of hits distribution.}
\label{fig:3}
\end{figure}


%%% Local Variables:
%%% mode: latex
%%% TeX-master: t
%%% End:
