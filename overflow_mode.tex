%%%%%%%%%%%%%%%%%%%%%%%%%%%%%%%%%%%%%%%%%%%%%%%%%%%%%%%%%%%%%%%%%%%%%%%%%%%%%%
\section{Run 281: ``ROC buffer overflow'' mode}
%%%%%%%%%%%%%%%%%%%%%%%%%%%%%%%%%%%%%%%%%%%%%%%%%%%%%%%%%%%%%%%%%%%%%%%%%%%%%%
The readout configuration of Run 281 had the event window size of 50 us
and the pulser rate of 60 kHz

\subsection{Hit timing and occupancy}\label{over}
The first distributions to look at are the time distributions of hits in 
different channels and the distribution of the total number of hits
in a given channel (occupancy) as a function of the channel number.
The timing distributions of hits in channel 0 of the first FPGA
and in channel 2 of the second FPGA are shown in Fig.\ref{fig:1}.
The left distribution is, as expected, uniform, however the right one looks
less trivial.

\begin{figure}[H]
  \hspace{-0.5in}
  \begin{tikzpicture}
    \node[anchor=south west,inner sep=0] at (0,0.) {
      % \node[shift={(0 cm,0.cm)},inner sep=0,rotate={90}] at (0,0) {}
      % \makebox[\textwidth][c] {
      \includegraphics[width=0.5\textwidth]{figures/pdf/figure_00007_timedistr_roc_simulation_ch0_281}
      % }
    };
    \node[anchor=south west,inner sep=0] at (10,0.) {
      % \node[shift={(0 cm,0.cm)},inner sep=0,rotate={90}] at (0,0) {}
      % \makebox[\textwidth][c] {
      \includegraphics[width=0.5\textwidth]{figures/pdf/figure_00003_timedistr_roc_simulation_ch2_281}
      % }
    };
  \end{tikzpicture}
  \caption{
    \label{fig:1}
    left:
    time distribution of hits in the channel 0, first FPGA
    right:
    time distribution of hits in the channel 2, second FPGA
    }
\end{figure}

The distributions in Fig.\ref{fig:1} are easier to understand by looking at the occupancy plot in Fig.\ref{fig:2}.left.
The channel ordering in this plot corresponds to the readout order.
Channels in the beginning of the readout sequence always have all their hits read out,
  however that is not true for the channels in the end of the readout sequence.
\begin{figure}[H]
  \hspace{-0.5in}
  \begin{tikzpicture}
    \node[anchor=south west,inner sep=0] at (0,0.) {
      % \node[shift={(0 cm,0.cm)},inner sep=0,rotate={90}] at (0,0) {}
      % \makebox[\textwidth][c] {
      \includegraphics[width=0.5\textwidth]{figures/pdf/figure_00004_nhitsvschannel_roc_simulation_281}
      % }
    };
    \node[anchor=south west,inner sep=0] at (10,0.) {
      % \node[shift={(0 cm,0.cm)},inner sep=0,rotate={90}] at (0,0) {}
      % \makebox[\textwidth][c] {
      \includegraphics[width=0.5\textwidth]{figures/pdf/figure_00014_nhitsvschannel_roc_simulation_281}
      % }
    };
  \end{tikzpicture}
  \caption{
    \label{fig:2}
    left: number of hits versus the channel number.
    The channels are numbered in the readout order;
    right: zoom on the last channels in the readout sequence. The data and MC distributions
    differ from each other by $\sim$ 10$^{-3}$.
  }
\end{figure}

Figure \ref{fig:66} shows the distribution of the number of hits in channel 0.
For the event window of 50 us and the time between the pulses of 16 us,
the number of hits could be 3 or 4,
depending on the timing offset of a given readout window with respect to the generated timing sequence.
This distribution plays a key role in understanding of the occupancy plot.
\begin{figure}[H]
\centering
\includegraphics[width =0.8\textwidth]{figures/pdf/figure_00066_nhits_ch00_run281.pdf}
\caption{
  The distribution of the number of hits in the channel 0 of the first digi FPGA. for run 281
}
\label{fig:66}
\end{figure}

In the distribution shown in Figure \ref{fig:2}.left,
the first 68 channels are the ones with 4 hits per channel in the first FPGA
and three hits per channel in the second FPGA, 
resulting in the total of 255 hits.
The second plateau extending from 68 to 75 corresponds to the channels
with 3 hits per channel in the first FPGA and 4 hits per channel in the second one.
  The ``dent'' in the end of the second plateau is due to the fact that the 48 channels of the first FPGA
  yield 144 hits, so the second FPGA contributes 111 hits. The first 27 channels of the second FPGA contribute
  4 hits per channel each, but as 111 is not an integer of 4, the three hits from channel 28 in the readout sequence
  fill up the total ROC buffer of 255 hits.
There is a big step at the end of this plateau, because if we count the number of hits
in the first FPGA we get 144, so in the second FPGA we have 111 hits in total,
due to the fact that the maximum number of hits in total is 255.
111 is not divisible by 4, so the first 27 channels in the second FPGA will have 4 hits
and the last one will have 3 hits.
The last plateau corresponds to events with 3 hits per channel from the first FPGA
and 3 hits per channel from the second FPGA.

A zoom on last channels is shown on the right picture of Fig.\ref{fig:2}.
The relative difference between the data and the MC distributions is at a level of $10^{-3}$,
which is a very good agreement.
Coming back to Fig.\ref{fig:1}, the first channels in the readout sequence
always have all their hits read out,
while the channels in the end of the readout sequence - do not,
as the ROC hit buffer gets filled up after
the first 255 hits are read out.
This results in a uniform time distribution for the first channels readout and in a non-uniform
time distribution for the last readout channels, depending on $T_{gen}$ and $T_{EW}$.
The dips in the hit timing distribution for channel 2 are defined by the timing offset
between the two FPGA pulsers. 


%%%%%%%%%%%%%%%%%%%%%%%%%%%%%%%%%%%%%%%%%%%%%%%%%%%%%%%%%%%%%%%%%%%%%%%%%%%%%%
\subsection{Number of hits}
Fig. \ref{fig:3} shows that in the ``buffer overflow'' mode all events,
as expected, have 255 hits read out per event.

\begin{figure}[!h]
\centering
\includegraphics[width =0.8\textwidth]{figures/pdf/figure_00008_nhits_281}
\caption{
  The distribution of the total number of hits read out per event
}
\label{fig:3}
\end{figure}


%%% Local Variables:
%%% mode: latex
%%% TeX-master: t
%%% End:
